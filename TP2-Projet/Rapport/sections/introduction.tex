\newpage 
\section{Introduction}
\subsection{Cas pratiques}
Ce projet est composé de deux cas pratiques distincts.
\subsubsection{Cas pratique 1}
Dans le premier cas pratique, il faut reprendre la To Do List du TP1. Elle va être complétée avec des options pour éditer les tâches et les supprimer.\\
L'édition se fait lors d'un clique sur l'une des tâches de la liste.\\
La suppression se fait avec un swipe ou à l'aide du bouton edit.\\
Une autre option à implémenter est de sauvegarder les tâches dans une base de donnée pour qu'elles soient rechargés lors du prochain lancement de l'application.

\subsubsection{Cas pratique 2}
Dans ce cas pratique, il va falloir réaliser une vue qui permet de lire une vidéo. Il faudra également implémenter le swipe pour changer de vue et une toolbar sur l'écran principal.\\ La partie concernant la liste de musique et sa lecture n'a pas pu être implémentée, car nous ne disposons pas d'IPhone et le simulateur ne permet pas de générer la liste de musiques.

\subsection{Buts}
\begin{enumerate}
	\item Se familiariser avec le langage swift
	\item Utiliser les identificateurs de segue
	\item Utilisation de « gesture »
	\item Créer une structure.
	\item Utiliser le protocole NSCoding.
	\item Utilisation de « MediaPlayer »
\end{enumerate}
