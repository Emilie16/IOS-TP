\section{Réponse aux questions}
\subsection{Question 1}
\textbf{Donnée: }Expliquez avec vos mots, quelle est l’utilité d’identifier la segue.\\\\
Dans le projet, on a utilisé des segues "identifiées" dans les méthodes preparForSegue. Cela a permis en fonction de quelle segue a déclenché l'action de passer ou non des informations à la prochaine vue.

\subsection{Question 2}
\textbf{Donnée: }Expliquez l’utilité d’une « unwind segue ».\\\\
Elle permet de quitter une vue et de revenir simplement à une autre en appelant une méthode spécifique de cette dernière. L'avantage est que l'on sait exactement à quel endroit du code on revient dans l'autre vue, cela permet de faire une action spécifique comme par exemple mettre à jour le contenu d'une TableView.

\subsection{Question 3}
\textbf{Donnée: }Expliquez avec vos mots, qu’est-ce qu’une structure.\\\\
Une structure est semblable à une énumération. Ses propriétés sont semblables aux classes. Le différence est que la struct fonctionne par copie alors que les classes utilisent des références.

\subsection{Question 4}
\textbf{Donnée: }Expliquez le principe de « Convenience » avec vos propres mots.\\\\
Le mot clé convenience est utilisé pour les initialisations de classes. Cela permet d'offrir à l'utilisateur plusieurs manière d'initialiser la classes en passant un nombre de paramètres variables et définir des valeurs par défaut. Voici un exemple plus parlant:
\begin{lstlisting}
init(content: String, sender: String, recipient: String)
{
	self.content = content
	//Rule 1:  Calling designated Initializer from immediate superclass
	super.init(sender: sender, recipient: recipient)
}
convenience init()
{
	//Rule 2:  Calling another initializer in same class
	self.init(content: "")
}
convenience init(content: String)
{
	//Rule 2:  Calling another initializer in same class
	self.init(content: content, sender: "Myself")
}
convenience init(content: String, sender: String)
{
	//Rule 2 and 3:  Calling the Designated Initializer in same class
	self.init(content: content, sender: sender, recipient: sender)
}
\end{lstlisting}